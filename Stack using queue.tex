\documentclass{article}
\usepackage{listings}
\usepackage{graphicx}

\title{Implementing a Stack Using Two Queues}
\author{B-09,15}
\date{\today}

\begin{document}
\maketitle

\section{Introduction}
A \textbf{Stack} is a fundamental data structure that follows the Last-In-First-Out (LIFO) principle. This means that the last element added to the stack is the first one to be removed. Stacks support two primary operations: \texttt{push} (to add an element) and \texttt{pop} (to remove the top element).

\section{Stack Using Two Queues}
To implement a stack using two queues, you can use the following approach:

- Maintain two queues, which we'll call \texttt{queue1} and \texttt{queue2}.

- The \texttt{push} operation can be achieved in \textit{linear time}. To push an element, enqueue it into \texttt{queue1}.

- The \texttt{pop} operation can be achieved in \textit{constant time}. To pop an element, dequeue all elements from \texttt{queue1} except the last one and enqueue them into \texttt{queue2}. The last element in \texttt{queue1} is the one to be popped.

\section{Time Complexity}
In this context:

- \textit{Linear time} means that the time complexity grows linearly with the size of the data structure (\(O(N)\)).

- \textit{Constant time} means that the time complexity remains constant (\(O(1)\)) regardless of the data size.

\section{Algorithm}
To construct a stack using two queues (q1, q2), we need to simulate the stack operations by using queue operations:

\textbf{Push}
\begin{itemize}
    \item if q1 is empty, enqueue E to q1
\end{itemize}
\begin{itemize}
    \item if q1 is not empty, enqueue all elements from q1 to q2, then enqueue E to q1, and enqueue all elements from q2 back to q1
\end{itemize}
\textbf{Pop}
\begin{itemize}
    \item dequeue an element from q1
\end{itemize}
As we see, q1 acts as the main source for the stack, while q2 is just a helper queue that we use to preserve the order expected by the stack.

The pseudocode of the push and pop operations are:
\begin{figure}
    \centering
    \includegraphics[width=0.75\linewidth]{Stack-Operations-768x475.png}
    \caption{Enter Caption}
    \label{fig:enter-label}
\end{figure}
\section{Program Implementation}
\lstset{language=C}
\begin{lstlisting}
// C code for implementing a stack using two queues
#include <stdio.h>
#include <stdlib.h>

struct Queue {
    // Define the structure for a queue
    // ...

    // Function to enqueue an element
    void enqueue(int item) {
        // Code here...
    }

    // Function to dequeue an element
    int dequeue() {
        // Code here...
    }
};

struct StackUsingQueues {
    // Define the structure for a stack using two queues
    struct Queue queue1;
    struct Queue queue2;

    // Function to push an element onto the stack
    void push(int item) {
        // Code here...
    }

    // Function to pop the top element from the stack
    int pop() {
        // Code here...
    }
};
\end{lstlisting}

\section{Program Output}
Show the expected output when running the implemented stack program. Include sample inputs and outputs.
\begin{figure}
    \centering
    \includegraphics[width=1\linewidth]{PushanelementinStack.png}
    \caption{Enter Caption}
    \label{fig:enter-label}
\end{figure}
\section{Conclusion}
In this project, we demonstrated how to implement a stack using two queues with specific time complexities. Achieving linear time for \texttt{push} and constant time for \texttt{pop} operations is a significant accomplishment in data structure design.

\end{document}